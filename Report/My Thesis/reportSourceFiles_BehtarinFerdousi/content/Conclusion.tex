\chapter{Conclusions} \label{ch:conclusion}
A demand for intelligent collision avoidance systems is timeless. State estimation algorithms can be used to track vehicles with assurance to predict a collision-free path. One significant factor is to fit the behavior of traffic participants to a mathematical model. Besides, the higher the number of measured states, the sharper is the enclosure. In comparison to the segment intersection techniques, the H-$\infty$-based interval observer does not worsen much with model complexity. This makes the interval observer suitable to be used for complex models. In addition, the H-$\infty$-based inverval observer is the fastest estimator, and its estimated bounds can be tuned using parameters. In conclusion, the H-$\infty$-based interval observer along with the point-mass model is the best choice for automobile collision avoidance systems. This thesis can be a starting point to review realistic higher-defined models of the tracked vehicle and compare the performance of the state estimation methods. The state estimation methods can further be evaluated by the effect of the initial estimated state. Further developments can be to add non-linear state-estimation algorithms on complex vehicle models in the comparison.