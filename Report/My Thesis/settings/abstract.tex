\chapter{\abstractname}
With the current rate of development in autonomous vehicles, the demand for a high-intelligent collision avoidance system is increasing. Due to the inability to determine the inner state of tracked vehicles from Lidar, GPS (Global Positioning System), and radar sensors, researchers have utilized state estimation methods to converge available measurements to the true state of the system. Set-based methods are used to enclose the true state of the system in a set, in contrast to stochastic methods which give a point-estimate close to the true state. Encapsulating the true state in a set is important to not allow any divergence from the true state for safety-critical tasks in autonomous vehicles. The purpose of this thesis is to review and implement different algorithms of set-based state estimation, using zonotopes as domain representation, on existing datasets of real traffic participants (approx. 10,518 entities). The algorithms implemented are segment intersection methods (using F-radius, P-radius, and volume) and an interval observer (using H-$\infty$ observer). They are compared in terms of computation time, time to converge, tightness of bound and accuracy. The H-$\infty$ interval observer performs better in terms of computation time but starts with a wider initial bound. Segment intersection minimization using P-radius is faster than using F-radius, but compromises on the bounds and accuracy. Of all the methods compared, segment minimization using F-radius gives the most desirable estimates for this use-case.