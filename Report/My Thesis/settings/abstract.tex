\chapter{\abstractname}
With the current rate of development in autonomous vehicles, the demand for a high-intelligent collision avoidance system is increasing. Due to the inability to determine the inner state of tracked vehicles from Lidar, GPS (Global Positioning System), and radar sensors, researchers have utilized state estimation methods to converge available measurements to the true state of the system. Set-based methods are used to enclose the true state of the system in a set, in contrast to stochastic methods which give a point-estimate close to the true state. Encapsulating the true state in a set is important to not allow any divergence from the true state for safety-critical tasks in autonomous vehicles. The purpose of this thesis is to review and implement multiple set-based state estimation algorithms, using zonotopes as domain representation, on existing datasets of real traffic participants (approx. 10,518 entities). The algorithms implemented are segment intersection methods (optimizing the F-radius, P-radius, and volume) and an interval observer (using H-$\infty$ observer). They are compared in terms of computation time, time to converge, tightness of bound, and accuracy. Furthermore, because the choice of vehicle dynamic model impacts the performance, the model is altered among constant velocity, constant acceleration, and the point-mass model to compare the influence. Among the estimators, the H-$\infty$-based interval observer is the fastest and the most accurate estimator. However, its performance depends on the initial estimation error bounds, computation of which is not challenging for a collision avoidance system. To summarize, the H-$\infty$-based interval observer with the point-mass model performs the best to locate and track vehicles for a collision avoidance system.