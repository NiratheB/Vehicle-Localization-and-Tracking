\chapter{Conclusion} \label{ch:conclusion}
A demand for intelligent collision avoidance system is timeless. To take the load off sensors and hardware of a vehicle, state estimation algorithms can be used to track vehicles and estimate properties required for collision-free path prediction. On comparing multiple techniques using different models to represent the tracked vehicle, it can be concluded that the segment intersection minimizing F-radius ensures faster convergence to more accurate and tighter bounded estimation. H-$\infty$ and P-radius carry out off-line computation and hence are ahead in terms of run-time computation cost; nonetheless, as a consequence, these methods over-approximate and do not improve estimation significantly for each measurement. The choice of a model to represent the state of the system also has a significant effect on the performance. To estimate velocity, the constant velocity model gives better results, whereas, for acceleration, the point-mass model gives a better estimate compared to the constant acceleration model. This paper can be a starting point to implement higher-defined models of the tracked vehicle and compare the performance of state estimation methods. The state estimation methods can further be evaluated on implementing with a distinguishable set of initial starting states to determine the effect of the initial estimated state on the algorithms if any. Further developments can be to use non-linear state-estimation algorithms on complex vehicle models and compare the performance.