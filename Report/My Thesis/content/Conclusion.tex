\chapter{Conclusion} \label{ch:conclusion}
A demand for intelligent collision avoidance system is timeless. To take load off sensors and hardware of a vehicle, state estimation algorithms can be used to track vehicle and estimate properties required for collision-free path prediction. On comparing multiple techniques using different models to represent the tracked vehicle, it can be concluded that the segment minimization using F-radius gives a good trade-off between fast convergence, accuracy and tight bound. H-$\infty$ and segment minimization using P-radius are ahead in terms of computation time because these methods carry out computation off-line before any input measurement, and hence as a consequence over-approximates and does not improvise estimation significantly for each measurement. Choice of model to represent the state of the system also has significant effect on the performance. To estimate velocity, constant velocity model gives better results, whereas for acceleration, the point mass model gives better estimate compared to the constant acceleration model. This paper can be a starting point to implement higher defined models of tracked vehicle and compare performance of state estimation methods. The state estimation methods can further be evaluated on implementing with distinguishable set of initial starting state to determine the effect of initial estimation on the algorithms, if any. Further developments can include implementing the technique in real vehicle system and using the estimation to track vehicles.