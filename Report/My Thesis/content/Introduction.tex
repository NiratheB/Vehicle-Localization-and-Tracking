\chapter{Introduction} \label{ch:intro}
There is a steep progress in research and development of autonomous vehicles. The race to the top of automobile industry, participated by companies like BMW, Tesla, Waymo/Google, requires fast devlopment and vigorous testing of novel technology. One of the many challenges of this field is to ensure collision avoidance. With no human behind wheels for Level 5 \cite{SAE2014} cars, the vehicle must keep track of roads, surrounding traffic participants (like vehicles and pedestrians) in different circumstances including rain and fog, to ensure safety of its passengers. Current collision avoidance systems based on sensors, radar and camera will be overwhelmed with high computation demands for this purpose. Tolerating error in such system can cause accidents; such error in vehicles have already caused real-life accidents, including one resulting in death \footnote{https://www.theguardian.com/technology/2018/mar/19/uber-self-driving-car-kills-woman-arizona-tempe}.


The Collision Avoidance System in a car system consists of two parts: Sensing and Tracking, and Maneuver. The sensing and tracking part is done by sensors like radar, camera and GPS (Global Positioning System). With advancement in technologies in image processing, image analysis and object detection and the decline in the cost of camera sensors, the sensing and tracking is developing fast. Although cameras can classify vehicles, it cannot gurantee measurement in low-light environment (e.g. night) \cite{Hirz2018}. On the other hand, radar gurantee robustness to weather in exchange of high cost. Similarly, GPS has disturbances too which favour methods using combination of sensors to cover each others\textquotesingle \: drawbacks to provide vehicle localization. After getting the data, the maneuver is carried out in a way to avoid collision with the location found from the sensors. The probable location of the tracked vehicle, is thus also important to calculate a predicted trajectory. However, all the data to predict the vehicle's location is not measurable using just sensors. Furthermore, the sensor data are not 100\% accurate, and hence solely cannot be relied to carry on maneuver to avoid collision.

Due to the lack of quality and availability of sensors, researchers have used state estimation algorithms to determine the state of the tracked vehicle. One of the widely applied technique is the Kalman filter, that requires a probability distribution of perturbation in the measurements. Such statistical data required for this method is not always deductible for all situations. Moreover, Kalman filter provides point estimation, not favourable for the automobile scenario. This motivates to use set-based state estimation methods.

The set-based state estimation technique provides a set of state bounding the true state of the system considering the bounds of the noise in the measurement and process. The skeleton of every set-based state estimation method is similar to Kalman filter, consisting of a prediction step and a correction step. The prediction step comprises of computing the state estimation using previous estimation and a model to define the transition. The correction step differs across algorithms and has been solved by Single Value Decomposition, computing Gain Matrix, solving LMI etc. Three distinguishable segment minimization methods and one Luenberger observer are implemented and evaluated in this paper. 

Comparing different domain representation of the sets enclosing the possible state of the system, zonotopes are chosen for this paper as opposed to ellipsoid and polytopes due to higher accuracy for a lower computation cost. Furthermore, zonotopes have gained fame for state estimation because of wrapping effect(i.e. not increasing in size in time due to accumulated noises) and Minkowski sum(i.e. sum of zonotopes is also a zonotope). We used CORA in Matlab\textsuperscript{\tiny\textregistered} for the functionalities in zonotope required for state estimation.

In order to utilize the state estimation algorithms, the foremost necessary step is to define the tracked vehicle in a linear model. Although there are complex models that can be used to represent a vehicle state \cite{Althoff}, not all can be used due to unavailability of measurements like wheelbase, velocity, etc. as it is unlikely to be acquired in run-time from tracked vehicle. Hence, the models used in this paper to compare are the simplest, yet complete enough to determine the properties of tracked vehicle for trajectory prediction : Constant Velocity, Constant Acceleration and the Point Mass Model.

A high degree of accuracy and guarantee is necessity of the collision avoidance system, hence we chose to compare the set based state estimation algorithms for different scenarios involving dynamic traffic participants from a dataset collected from intersections using drone and fixed camera. A similar comparison can be found in \cite{Rath} on simulated data.

The paper is organized as follows. Chapter 2 presents the vehicle localization problem to be solved by state estimation algorithms. The following chapter 3 discusses the zonotope-based state estimation algorithms to be compared. Chapter 4 gives the evaluation of the algorithms, with extended results on chapter 6. Finally, chapter 5 concludes with a summary and a discussion of possible future works.




