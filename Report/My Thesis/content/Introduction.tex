\chapter{Introduction} \label{ch:intro}
There is a steep progress in research and development of autonomous vehicles. The race to the top of automobile industry, participated by companies like BMW, Tesla, Waymo/Google, requires fast devlopment and vigorous testing of novel technology. One of the many challenges of this field is to ensure collision avoidance. With no human behind wheels for Level 5 \cite{SAE2014} cars, the vehicle must keep track of roads, surrounding traffic participants, like vehicles and pedestrians, in different circumstances including rain and fog, to ensure safety of its passengers. Current collision avoidance systems based on sensors, radar and camera will be overwhelmed with high computation demands for this purpose. Tolerating error in such system can cause accidents; such error in vehicles have already caused real-life accidents, including one resulting in death \footnote{https://www.theguardian.com/technology/2018/mar/19/uber-self-driving-car-kills-woman-arizona-tempe}.


The Collision Avoidance System in car system consists of two parts: Sensing and Tracking, and Maneuver. The sensing and tracking part is done by sensors like radar, camera and GPS (Global Positioning System). With advancement in technologies in image processing, image analysis and object detection and the decline in the cost of camera sensors, the sensing and tracking is developing fast. Although cameras can classify vehicles, it cannot gurantee measurement in low-light environment (e.g. night) \cite{Hirz2018}. On the other hand, radar gurantee robustness to weather with in exchange of high cost. Similarly, GPS has disturbances too which makes methods using combination of sensors to cover each others drawbacks to provide vehicle localization. After getting the data, the Maneuver is carried out in a way to avoid collision with the location found from the sensors. The probable location of the tracked vehicle, is thus also important to calculate a predicted trajectory. However, all the data to predict the vehicle's location is not measurable using just sensors and nevertheless the sensor data are not 100\% accurate, and hence solely cannot be used to carry on maneuver to avoid collision.

Due to the lack of quality and availability of sensors, researchers have used state estimation algorithms to determine the state of the tracked vehicle. One of the widely applied technique is the Kalman Filter, that requires a probability distribution of perturbation in the measurements.  

On the other hand, there exists set-based state estimation which provides a set of possible states of the system, instead of a close estimation like Kalman filter. A high degree of accuracy and guarantee is demanding in the collision avoidance system, hence we chose to compare the set based state estimation algorithms for the scenario. A similar comparison can be found in \cite{Rath}.

Comparing different domain representation of the sets enclosing the possible state of the system, we chose zonotopes as opposed to ellipsoid and polytopes due to higher accuracy for a lower computation cost. Furthermore, zonotopes have gained fame for state estimation because of wrapping effect(i.e. not increasing in size in time due to accumulated noises) and Minkowski sum(i.e. sum of zonotopes is also a zonotope). We used CORA in Matlab for the functionalities in zonotope required for state estimation.

In order to utilize the state estimation algorithms, the foremost necessary step is to define the tracked vehicle in a linear model. Although there are complex models that can be used to represent a vehicle state \cite{Althoff}, not all can be used due to unaivailability of measurements like wheelbase, velocity, etc. unlikely to be acquired in run-time from tracked vehicle. Hence, the models used in this paper to compare are the simplest : Constant Velocity, Constant Acceleration and the Singer Acceleration Model.

The paper is organized as follows. Chapter 2 presents the vehicle localization problem suitable to be solved by state estimation algorithms. The following Chapter 3 discusses the zonotope-based state estimation algorithms to be compared. Chapter 4 and 5 give the evaluation of the algorithms. Finally, chapter 6 concludes with a summary and a discussion of possible future works.




